\documentclass{article}
\usepackage[utf8]{inputenc}

\title{Probability Project report}
\author{ Bahzad Ahmed bb05083, Aaron Soares, Owais bin Asad }
\date{June 2020}

\begin{document}

\maketitle

\section{Introduction}
This report aims to provide an explanation of the solutions given in the probability project submitted by this group. Moreover, it also aims to help visualise the simulation using plots and graphs.

\section{Explanation}
\subsection{Task 3}
\subsubsection{Re-Entry Model}
The Re-entry model works by detecting on every step if the next step taken would cross the boundary, if so the particle stays where it is for the step, however it is allow to move in the next step if it isn't crossing the boundary. In order to achieve this we added a check for euclidean distance from the origin and compared with the radius every time a step is taken and work along that. This Re-entry model ensures that the particle does not escape the circle and that it can still move by changing its direction in the upcoming steps.\\
Our choice using such a model was such that we can ensure every step taken is allowed and random.

\subsection{Task 8}


\end{document}
