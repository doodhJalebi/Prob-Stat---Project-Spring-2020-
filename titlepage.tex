\documentclass[10pt, a4paper]{article}

\usepackage{graphicx}
\usepackage{amsmath}

\begin{document}
    
\title{MATH 310: Probability \& Statistics \\Project Report}
\author{Owais Bin Asad, Aaron Lucas Soares, Bahzad Ahmad}

\maketitle
\newpage

\section*{Task 1}
\subsection*{Simulation Model:}
In order to determine a good average distance from starting position for each step number, the simulation was run
100 times, each one being 1000 steps long. Refer to figure 1.

\subsection*{Mathematical Model:}
\begin{align*}
    y_{t+1} \approx y_{t} + X
\end{align*}

where X is a discrete random variable that takes on discrete values $[-1, 1]$ with probabilites $[0.5, 0.5]$ respectively.
$y_t$ is the distance from the starting point at time $t$. $y_{t+1}$ is the distance from the starting point at time $t+1$.


\section*{Task 2}
\subsection*{Simulation Model:}
In order to determine a good average number of steps for a fixed starting distance between the two persons, the simulation was run
100 times, each one being 1000 steps long, for each value of starting distance $x$. Refer to figure 2.

\subsection*{Mathematical Model:}
\begin{align*}
    E[T] = \frac{1}{2}(\text{time for person A to reach }\frac{d}{2})\text{ for even d}\\
    E[T] = \frac{1}{2}(\text{time for person A to reach }\frac{d+1}{2})\text{ for odd d}\\
\end{align*}

where T is a random variable that denotes the time taken in number of steps for person A and B to meet.
$d$ is the distance between them at time $t = 0$.


\section*{Task 4}
\subsection*{Simulation Model:}
This builds on Task 1's simulation model with the only difference being that the step size is modeled by a uniform random variable now.
Refer to figure 4.

\subsection*{Mathematical Model:}
\begin{align*}
    y_{t+1} \approx y_{t} + X
\end{align*}

where X is a continuous random variable that takes on values $[0, 1]$ with unifrom probabilites.
$y_t$ is the distance from the starting point at time $t$. $y_{t+1}$ is the distance from the starting point at time $t+1$.


\end{document}